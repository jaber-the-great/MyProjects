\chapter{ روش\nf های پیاده\nf سازی}
\setlatintextfont{Times New Roman}

\section{معرّفی روش\nf های پیاده\nf سازی}

در این فصل، انواع روش\nf های ممکن برای پیاده\nf سازی \lr{IMS} توسط پروژه \lr{clearwater}، به\nf طور خلاصه بیان شده\nf اند. مطالب این فصل بر اساس \cite{webcw} نوشته شده\nf اند. انواع روش\nf های پیاده\nf سازی عبارتند از :
\begin{itemize}
\item پیاده\nf سازی  یکپارچه و در مقیاس کوچک 
\begin{enumerate}
\item نصب بر روی ابر محاسباتی منعطف آمازون (\lr{Amazon EC2})
\item نصب خودکار بر روی سکوهای مجازی\nf سازی
\item نصب دستی بر روی سکّوهای مجازی\nf سازی  
\end{enumerate}
\item پیاده\nf سازی توزیع\nf شده و در مقیاس بزرگ
\begin{enumerate}
\item  نصب خودکار بر روی  \lr{Amazon EC2}
\item  نصب دستی سرورها بر روی ماشین\nf های مجزّا 
\end{enumerate}
\item نصب با استفاده از کد منبع
\end{itemize}


\section{پیاده\nf سازی یکپارچه در مقیاس کوچک }
\subsection{معرّفی}
با وجود این که {\lr{claerwater} طوری طرّاحی شده است که قابلیت مقیاس\nf پذیری در ابعاد وسیع را دارد، اما این امکان وجود دارد که بتوان تمام اجزای آن را روی یک رایانه\nf ی شخصی نصب کرد. این روش نصب، نسبت به سایر روش\nf ها ساده\nf تر است اما عملکرد محدودی نسبت به آن\nf ها دارد. با وجود عملکرد محدود، پیاده\nf سازی \lr{clearwater} در مقیاس کوچک باعث آشنایی با معماری، المان\nf ها و پیکربندی \lr{clearwater} می\nf شود. این کار باعث می\nf شود که بتوانید عملکرد بهتر و آگاهانه\nf تری در پیاده\nf سازی مقیاس بزرگ با استفاده از روش\nf های توزیع\nf شده داشته باشید.

\subsection{انواع روش\nf های یکپارچه}
در تمام روش\nf های یکپارچه، از  \lr{All-In-One Image} استفاده می\nf شود.  \lr{All-In-One Image} شامل موارد زیر است:
\begin{itemize}
\item سیستم عامل اوبونتو \lr{14.04}، پیکربندی\nf شده برای استفاده از \lr{DHCP}
\item المان\nf های \lr{Bono, Sprout, Homestead, Homer, Ellis}
\item اسکریپت\nf های پیکربندی خودکار \lr{clearwater}
\end{itemize}

\label{yekparchePart}

در هنگام راه\nf اندازی، \lr{image} از طریق \lr{DHCP} پیکربندی \lr{IP} خود را به\nf دست می\nf آورد. سپس، اسکریپت پیکربندی خودکار، المان\nf های \lr{BONO, Sprout, Homestead, Homer} و \lr{Ellis} را مطابق با \lr{IP} به\nf دست\nf آمده، پیکربندی می\nf کند. این \lr{image} طوری طرّاحی شده است که روی یک ماشین مجازی با پردازنده\nf ی تک\nf هسته\nf ای، 2 گیگابایت \lr{RAM}، و 8 گیگابایت حافظه\nf ی دیسک سخت قابل اجرا است. در صورت استفاده از ابر محاسباتی منعطف آمازون، مدل \lr{t2.small}، این حدّاقل موارد مورد نیاز را فراهم می\nf کند. 

\subsubsection{نصب بر روی ابر محاسباتی منعطف آمازون}

می\nf توان \lr{All-In-One Image} را بر روی ابر محاسباتی منعطف آمازون نصب کرد. برای نصب ابتدا باید سرویس مورد نظر خود را که از لحاظ امکانات سخت\nf افزاری، حدّاقل کارایی موردنیاز برای نصب \lr{All-In-One Image} دارد(مانند \lr{t2.small}) را خریداری کنید. نحوه\nf ی نصب بر روی ابرمحاسباتی منعطف آمازون در \cite{webec2} بیان شده است. 


\subsubsection{نصب خودکار بر روی سکّوی مجازی\nf سازی}

سکّوی مجازی\nf سازی مورد استفاده در این روش، باید قابلیت پشتیبانی از \lr{OVF}\RTLfootnote{سرواژه\nf ی عبارت \lr{Open Virtualization Format} و به معنای فرمت مجازی\nf سازی باز. \lr{OVF}، یک استاندارد متن\nf باز برای بسته\nf بندی و توزیع نرم\nf افزارهای ماشین مجازی است.} را داشته باشد. تمامی سکّوهای مجازی\nf سازی که از این قابلیت پشتیبانی می\nf کنند، می\nf توانند برای نصب مورد استفاده قرار گیرند؛ تاکنون، سکّوهای زیر تست شده\nf اند و این تست، موفّقیّت\nf آمیز بوده\nf است.
\begin{latin}
\begin{itemize}
\item VMware Player
\item VirtualBox
\item VMware ESXi
\end{itemize}
\end{latin}
از آنجایی\nf که  \lr{All-In-One Image} برای به\nf دست آوردن پیکربندی \lr{IP} خود از \lr{DHCP} استفاده می\nf کند، سکّوی مجازی\nf سازی موردنظر باید یا در نهاد خود \lr{DHCP} را به\nf کار برده باشد و یا به یک شبکه\nf ی حاوی سرور \lr{DHCP} متّصل باشد. مراحل نصب در این روش، در بخش \ref{cwaio} توضیح داده شده\nf اند.

\subsubsection{نصب بر روی سکّوی مجازی\nf سازی با استفاده از کد منبع}

اگر سکّوی مجازی\nf سازی مورد استفاده، \lr{OVF} را پشتیبانی نمی\nf کند، این امکان وجود دارد که نود \lr{All-In-One} را به\nf صورت دستی \lr{build} کرد. برای این کار، ابتدا باید سیستم عامل اوبونتو \lr{14.04} نسخه\nf ی سرور را بر روی ماشین مجازی نصب کرد. سپس باید فایل اسکریپت مورد نیاز برای انجام تغییرات لازم، از \lr{GitHub} دریافت شود. برای دریافت این فایل به \cite{gitmanual} مراجعه شود.

با اجرای این اسکریپت، المان\nf های \lr{clearwater} و رابط\nf های آن\nf ها بر روی ماشین مجازی نصب خواهندشد و سپس، \lr{All-In-One Image} ساخته خواهدشد. با راه\nf اندازی مجدّد ماشین مجازی، \lr{All-In-One Image} اجرا شده و مابقی کارها مانند برقراری تماس تلفنی، مشابه روش استفاده از فایل \lr{OVF} در پیاده\nf سازی یکپارچه است. همچنین، برای اطّلاعات بیشتر در مورد نحوه\nf ی اجرای اسکریپت و انجام تغییرات لازم به \cite{webmanual} مراجعه شود.


\subsection{قابلیت\nf ها و محدودیّت\nf ها}
از آنجایی\nf که \lr{All-In-One Image} شامل تمام اجزای معماری \lr{clearwater} می\nf باشد، می\nf توان گفت که تقریباً تمام قابلیت\nf های اصلی یک نصب توزیع\nf شده و در مقیاس بزرگ(مانند برقراری تماس تلفنی و تنظیمات مربوط به شبکه و ...) را دارد. محدودیّت\nf های اصلی این روش، به شرح زیر می\nf باشند.

\subsubsection{عملکرد و کارایی ضعیف}
از آن\nf جایی\nf که تمام نرم\nf افزارها و اجزاء در روش پیاده\nf سازی یکپارچه، روی یک ماشین مجازی نصب و اجرا می\nf شوند، عملکرد و کارایی سیستم در این روش نسبت به روش توزیع\nf شده ضعیف\nf تر است. از این رو، با استفاده از این روش پیاده\nf سازی تنها قادر به ارائه سرویس به تعداد کاربران محدودی خواهیم بود. 

\subsubsection{عدم وجود قابلیت مقیاس\nf پذیری}
اصلی\nf ترین محدودیّت این روش، عدم امکان مقیاس\nf کردن است. در واقع، هیچ راه و گزینه\nf ای وجود ندارد که بتوانیم به\nf وسیله\nf ی آن، این نوع پیاده\nf سازی  را مقیاس کنیم و ماشین\nf های مجازی بیشتری اضافه کرده تا ظرفیت سیستم را ارتقاء دهیم.

\subsubsection{عدم امکان ایجاد افزونگی}

سرویس سنّتی ارتباطات، برای دستیابی به قابلیت اطمینان از روش تکرار و انعکاس داده\nf ها استفاده می\nf کند که غالباً هم به\nf وسیله\nf ی طرّاحی یک\nf به\nf یک پیاده\nf سازی شده\nf اند. این روش پرهزینه و پیچیده را نمی\nf توان به\nf خوبی در محیط ابری یا مجازی\nf سازی\nf شده به\nf کار برد. در \lr{clearwater}، این کار از طریق ایجاد یک خوشه از یک جزء و افزونگی\nf هایش، و هم\nf چنین از طریق ایجاد افزونگی جغرافیایی صورت می\nf گیرد. امکان ایجاد افزونگی در روش پیاده\nf سازی در مقیاس کوچک وجود ندارد.

از آن\nf جایی\nf که در این روش، کل اجزاء روی یک ماشین مجازی نصب و اجرا می\nf شوند، اگر این ماشین مجازی از کار بیافتد و یا به\nf درستی کار نکند، کل سیستم از کار می\nf افتد. در صورتی که در روش\nf های توزیع\nf شده، خرابی یک بخش، تنها باعث از کار افتادن بخشی از سرویس می\nf شود؛ وجود افزونگی\nf ها نیز باعث می\nf شود که با از کار افتادن بخشی از سیستم بتوانیم بدون به\nf وجود آمدن اشکال و اختلال، سرویس خود را ارئه کنیم.  

\subsubsection{ناحیه سخت-رمزی }

سخت-رمزی \RTLfootnote{\lr{Hard-coded}} یک روش توسعه است که در آن ورود و پیکربندی داده\nf ها به\nf صورت مستیقیم در درون کد منبع جاسازی شده است. لذا، توسعه\nf دهندگان دیگر نمی\nf توانند به\nf راحتی آن را تغییر دهند. در روش پیاده\nf سازی یکپارچه، از یک ناحیه سخت-رمزی به نام \lr{example.com} استفاده می\nf شود. بنابراین، \lr{SIP URI} کاربران مشابه  \lr{sip:6505551234@example.com} خواهدبود(بخش \ref{cwaio}). به\nf صورت پیش\nf فرض، \rl{SIP} از این ناحیه برای مسیریابی استفاده می\nf کند اما \rl{example.com} میزبان شما را به\nf دست  نمی\nf آورد؛ بنابراین نیاز داریم که برای هر کاربرِ \rl{SIP}، یک وکیل خارج باند\RTLfootnote{\lr{Outbound proxy}} پیکربندی کنیم. این پیکربندی، در اپلیکیشن \lr{SIP} سمت کاربر انجام می\nf شود.

\section{پیاده\nf سازی در مقیاس بزرگ و پیاده سازی با کد منبع}

پیاده\nf سازی در مقیاس بزرگ، برای ارائه\nf ی سرویس به تعداد زیادی کاربر استفاده می\nf شود. این کار که بیشتر جنبه\nf ی تجاری دارد، به سه روش قابل انجام است. در روش اوّل از ابر آمازون به عنوان بستر پیاده\nf سازی استفاده می\nf شود. در این روش، فقط با استفاده از یک دستور می\nf توان  پیاده\nf سازی کامل \lr{clearwater} در ابعاد(مقیاس) دلخواه را ایجاد کرد. لذا، راحت\nf ترین روش برای پیاده\nf سازی در مقیاس بزرگ است.

در روش دوّم که به آن روش نصب دستی گفته\nf می\nf شود، باید بسته\nf های دبیان\RTLfootnote{\lr{Debian}} را نصب کرده و هر ماشین را به\nf طور جداگانه پیکربندی کرد. پیاده\nf سازی این روش، نیازمند حدّاقل ۶ ماشین است. بر روی هر یک از این ماشین\nf ها، باید نسخه\nf ی سرور سیستم عامل اوبونتو \lr{14.04} نصب شود. سپس، هر یک از نودهای معماری \lr{clearwater} بر روی یک ماشین نصب می\nf شود. پس از نصب نودها، تنظیمات مربوط به دیوار آتش و سرویس نام دامنه باید به\nf صورت دستی انجام شوند.

در صورتی که سیستم عامل استفاده\nf شده، مبتنی بر اوبونتو نیست و یا نیاز است که تغییراتی در کدهای منبع اعمال شود، می\nf توان از روش سوّم استفاده کرد. در این روش، باید با استفاده از کدهای منبع \lr{clearwater}، هر یک از اجزای معماری آن را جداگانه \lr{build} کرد. دستورالعمل استفاده از هر یک از اجزای معماری توسّط \lr{clearwater} منتشر شده است و با کمک این دستورالعمل\nf ها، می\nf توان این سیستم را از طریق کد منبع پیاده\nf سازی کرد.