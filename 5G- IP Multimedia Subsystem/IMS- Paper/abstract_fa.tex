\thispagestyle{empty}
\noindent
%\newcommand{\nf}{\char\value{226}\char\value{128}\char\value{140}}
\newcommand*{\nf}{^^^^200c}
\centerline{\textbf{\large{چکیده}}} \\
زیرسیستم چندرسانه\nf ای مبتنی بر پروتکل اینترنت، یک معماری برای انتقال داده\nf ی چندرسانه\nf ای(مانند صدا و تصویر) بر روی شبکه اینترنت می\nf باشد. این معماری که به آن \lr{IMS} گفته می\nf شود، قرار است جایگزین سوئیچ\nf های مداری در شبکه سلولی شود. مزیّت اصلی این سیستم نسبت به سیستم\nf های کنونیِ صدا بر روی \lr{IP}، فراهم کردن کیفیت سرویس(\rl{QoS}) برای ارتباطات تعاملی نظیر تماس صوتی و ویدیوئی است. پس از انجام تحقیقات در مورد نحوه پیاده\nf سازی، استانداردها، کاربردها و قابلیت\nf های \lr{IMS}، این سیستم به\nf وسیله \lr{clearwater} پیاده\nf سازی شد. \lr{claerwater}، یک پروژه متن\nf باز است که هسته \lr{IMS} را پیاده\nf سازی کرده است. با استفاده از \lr{clearwater}، سیستم \lr{IMS} به صورت یکپارچه و در مقیاس کوچک پیاده\nf سازی شد. سپس، با استفاده از \lr{IMS} و در بستر اینترنت، تماس تلفنیِ با کیفیت بالا بر روی \lr{IP} برقرار شد و کارکرد قسمت\nf های مختلف این سیستم مورد آزمایش قرار گرفت. در مرحله\nf ی بعد، کیفیت سرویس و سیگنالینگ\nf های مورد استفاده در \lr{IMS} و سیستم\nf های متداول صدا بر روی \lr{IP}، مورد مقایسه قرار گرفتند. همچنین، \lr{IMS} به شبکه\nf ی سلولی نسل چهارم(\lr{4G}) متّصل شد. با انجام این کار، تماس تلفنی در بستر شبکه\nf ی سلولی و از طریق \lr{IMS} برقرار شد. در مرحله\nf ی آخر نیز سعی شد که ارتباط بین \lr{IMS} و شبکه\nf ی تلفن ثابت برقرار شود تا کاربران تلفن ثابت و \lr{IMS} بتوانند با یکدیگر ارتباط تلفنی برقرار کنند. با توجّه به عدم دسترسی به تجهیزات مورد نیاز در زمان مناسب، انجام این کار میسّر نگردید.
 

\textbf{کلمات کلیدی:} \
\textit{زیرسیستم چندرسانه\nf ای، شبکه سلولی، کیفیت سرویس، \lr{IMS}}