\chapter{نتیجه\nf  گیری}

%با گسترش اینترنت و استفاده از اپلیکیشن\nf های پیام\nf رسان و اپلیکشن\nf های مکالمه\nf ی صوتی و ویدیوئی در بستر اینترنت، ارائه\nf ی سرویس اینترنت به نقش اصلی اپراتورهای تلفن همراه تبدیل شده است. ارائه\nf ی سرویس اینترنت به\nf جای سرویس\nf هایی نظیر پیامک و تماس صوتی، باعث کاهش سودآوری اپراتورهای تلفن همراه شده است. با توجّه به تحقیقات و بررسی\nf های صورت گرفته در مورد سیستم \lr{IMS} و دلایل پیدایش آن، می\nf توان گفت که راه\nf اندازی و استفاده از این سیستم، به زودی توسّط اپراتورهای تلفن همراه انجام خواهدشد. در شرایطی که اپراتورهای تلفن همراه تمایلی به استفاده از این سیستم نداشته\nf باشند، کاهش سودآوری، استانداردهای تدوین\nf شده و بازار رقابتی، آن\nf ها را مجبور به راه\nf اندازی و استفاده از \lr{IMS} خواهدکرد. استفاده\nf ی اپراتورهای تلفن همراه از سیستم \lr{IMS} باعث افزایش سودآوری آن\nf ها خواهد شد.

تحقیقات انجام\nf گرفته در این پروژه در مورد معماری، دلایل پیدایش، مزایا و کاربردهای \lr{IMS}، باعث ترقیب اپراتورهای تلفن همراه برای استفاده از این سیستم و همچنین هموار شدن مسیر برای راه\nf اندازی و استفاده آن می\nf شود. پیاده\nf سازی \lr{IMS} توسّط پروژه\nf ی متن\nf باز \lr{clearwater} در این پروژه، می\nf تواند اثباتی بر این موضوع باشد که می\nf توان این سیستم را به\nf صورت بومی و بدون نیاز به کشورهای خارجی راه\nf اندازی کرد. وجود روش\nf های پیاده\nf سازی در مقیاس بزرگ و همچنین تجربیات به\nf دست آمده از پیاده\nf سازی این سیستم در مقیاس کوچک، این امکان را به\nf وجود می\nf آورد که بتوان سرویس \lr{IMS} را در آینده\nf ای نزدیک و در مقیاس بزرگ راه\nf اندازی کرد. با این کار، علاوه بر کاهش هزینه\nf های اپراتورها، سودآوری آن\nf ها از طریق ارائه\nf ی سرویس\nf های نوظهور افزایش خواهدیافت و کاربران تلفن همراه از این سرویس\nf ها بهره\nf مند خواهندشد. راه\nf اندازی \lr{IMS}، امکان ورود شرکت\nf های شخص ثالث به بازار مخابرات را نیز فراهم می\nf کند.

در این پروژه، پس از راه\nf اندازی شبکه\nf ی \lr{4G} و \lr{IMS} به وسیله\nf ی پیاده\nf سازی\nf های متن\nf باز، ارتباط این دو سیستم با یکدیگر برقرار گردید. سیستم \lr{IMS} به\nf عنوان یک ارائه\nf دهنده\nf ی سرویس صدا بر روی \lr{IP} به شبکه\nf ی \lr{4G} متّصل شد و برقراری تماس صوتی نیز از طریق سیم\nf کارت و به\nf وسیله\nf ی \lr{IMS} صورت گرفت. نرم\nf افزاری شدن هسته\nf ی شبکه\nf های سلولی و پیاده\nf سازی نرم\nf افزاری \lr{IMS}، باعث عدم وابستگی اپراتورهای تلفن همراه به شرکت\nf های خارجی می\nf شود و تحریم در حوزه\nf ی مخابرات نمی\nf تواند باعث ایجاد مشکل در ارائه\nf ی سرویس اپراتورهای تلفن همراه شود.

اقدام به برقراری ارتباط \lr{IMS} با شبکه\nf ی تلفن ثابت، از کارهای دیگری است که در این پروژه صورت گرفت. به دلیل عدم دسترسی به\nf موقع به تجهیزات موردنیاز، این ارتباط برقرار نشد. کارهای انجام\nf شده در این زمینه، در پیوست (\ref{pstnPart}) آورده شده است. اتّصال \lr{IMS} به شبکه\nf ی تلفن ثابت و برقراری تماس تلفنی بین کاربران تلفن ثابت و کاربران \lr{IMS}، در آینده\nf ای نزدیک انجام خواهدگرفت. 

در این پروژه، کیفیت سرویس ارائه\nf شده توسّط \lr{IMS} و \lr{VOIP} مورد مقایسه قرار گرفت. با انجام این مقایسه، مشخّص گردید که در صورت استفاده از \lr{IMS} به عنوان ارائه\nf دهنده\nf ی سرویس \lr{VOIP} در بستر اینترنت و یا شبکه\nf ی محلّی، هر دو سیستم از قابلیت یکسانی در فراهم کردن کیفیت سرویس دارند. امّا در صورت اتّصال \lr{IMS} به شبکه\nf ی سلولی، این سیستم می\nf تواند کیفیت سرویس را برای کاربران شبکه\nf ی سلولی فراهم کند. با این وجود، \lr{IMS} این قابلیت را دارد که به عنوان یک ارائه\nf دهنده\nf ی سرویس \lr{VOIP} به حجم عظیمی از کاربران، مورد استفاده قرار گیرد.  

پیاده\nf سازی \lr{IMS} به\nf صورت توزیع\nf شده و در مقیاس بزرگ به عنوان سرور \lr{VOIP} دانشگاه صنعتی اصفهان از کارهای پیش رو است. دانشجویان و سایر دانشگاهیان که کاربران اصلی این سیستم هستند، می\nf توانند با استفاده از بستر شبکه\nf ی دانشگاه، از تماس تلفنی رایگان داخلی(در سطح دانشگاه) بهره\nf مند شوند. همچنین، با پیشرفت پروژه\nf ی \lr{IUT-LTE}\RTLfootnote{پروژه\nf ی پیاده\nf سازی نرم\nf افزاری شبکه\nf ی \lr{LTE} در دانشگاه صنعتی اصفهان}، دانشگاهیان می\nf توانند از طریق سیم\nf کارت، به هسته\nf ی این شبکه\nf ی سلولی متّصل شوند و توسّط \lr{IMS} تماس تلفنی برقرار کنند.
  
اتّصال \lr{IMS} از طریق \lr{PCRF} به شبکه\nf ی سلولی به\nf صورت جدّی پیگیری خواهدشد. با این کار، \lr{IMS} می\nf تواند با المان\nf های شبکه\nf ی سلولی تعامل کند و کیفیت سرویس را فراهم کند. همچنین، خدمات مالیِ حساب کاربران نیز از طریق \lr{PCRF} صورت می\nf گیرد. لذا با استفاده از آن، می\nf توان این سیستم را به\nf صورت تجاری عرضه کرد.

عدم وابستگی \lr{IMS} به شبکه\nf ی دسترسی، از دیگر مزایای این سیستم است که به کاربران امکان جابه\nf جایی بین شبکه\nf های دسترسی مختلف و همچنین برقراری تماس تلفنی از طریق دستگاه\nf های مختلف(نظیر تلفن همراه، رایانه\nf ی شخصی و ...) را می\nf دهد. این قابلیت، در کنار سرویس\nf های نوظهوری که توسّط \lr{IMS} ارائه می\nf شوند، باعث افزایش تمایل کاربران به استفاده از این سیستم می\nf شود و برای اپراتورهای تلفن همراه، مزیت رقابتی ایجاد می\nf کند.

ایجاد اپلیکیشن سرورهای جدید برای ارائه\nf ی سرویس\nf های نوظهور توسّط \lr{IMS}، از دیگر برنامه\nf هایی است که قرار است در دست کار قرار گیرد. ارائه\nf ی سرویس\nf هایی نظیر تماس ویدیوئی، کنفرانس صوتی و ویدیوئی، \lr{PoC} و \lr{IP messaging} ارزش افزوده\nf ی بالایی دارد. این سرویس\nf ها، می\nf توانند جایگزینی برای اپلیکیشن\nf های خارجیِ متداول نظیر \lr{Skype}، واتساپ و تلگرام باشند. با توجّه به استفاده\nf ی این سیستم از ترافیک داخلی کشور و عدم استفاده از سرورهایِ خارج از کشور، هزینه\nf ی استفاده از این سرویس نسبت به سرویس\nf های مشابه خارجی کمتر بوده و از امنیّت بالاتری برخوردار است.
