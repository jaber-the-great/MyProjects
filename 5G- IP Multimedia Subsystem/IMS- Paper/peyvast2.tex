\chapter{مسائل و مشکلات آزمایش \lr{clearwater}}
\label{LiveTestA}
\setlatintextfont{Times New Roman}
\subsubsection{دستور شماره ۱}
\begin{latin}
1. sudo apt-get install build-essential git  -\nf -yes
\end{latin}
دستور فوق که اوّلین دستور از مجموعه دستورات این بخش می\nf باشد، \lr{build-essential} را دانلود و بر روی سیستم نصب می\nf کند. \lr{build-essential}، یک \lr{meta package} است که شامل \lr{reference}\RTLfootnote{ارجاع} به تعداد زیادی بسته\RTLfootnote{\lr{Package}}می\nf باشد و این بسته\nf ها را بر روی سیستم نصب می\nf کند. این بسته\nf های نصب\nf شده، پیش\nf نیاز استفاده از سایر زبان\nf ها برنامه\nf نویسی در سیستم مورد نظر می\nf باشند. همچنین، هرگاه لازم باشد برنامه\nf ای از طریق کد منبع اجرا شود، باید این بسته\nf ها از قبل روی سیستم نصب شده باشند. 

در واقع، \lr{build essential} یک \lr{reference} به تمام بسته\nf های مورد نیاز برای کامپایل کردن یک نرم\nf افزار یا یک بسته دبیان است. این بسته\nf ها، به\nf طور معمول، شامل کامپایلرهای \lr{gcc/g++}، کتابخانه\nf های \lr{C} و \lr{C++} و یک سری ابزار دیگر می\nf باشند. با توجّه به مطالب گفته\nf شده، دستور فوق، بسته\nf های مورد نیاز برای استفاده از زبان \lr{Ruby} را بر روی سیستم ما نصب می\nf کند.



\subsubsection{دستور شماره 2}
\begin{latin}
\setlength{\parindent}{0ex}
2. curl -L http://get.rvm.io | bash -s stable
\end{latin}

دستور فوق، \lr{RVM}\RTLfootnote{\lr{Ruby Version Manager}} را روی سیستم نصب می\nf کند. \lr{RVM} یک ابزار خط فرمان است که امکان نصب آسان، مدیریت و کار با \lr{Ruby} را فراهم می\nf کند. به احتمال زیاد، به دلیل عدم وجود کلید \lr{gpg}\RTLfootnote{\lr{GNU Privacy Guard}: برای رمزنگاری داده\nf های  ارسالی بین سرور و کلاینت استفاده می\nf شود}
 بر روی سیستم، دستور فوق به\nf درستی انجام نشود. برای ایجاد یا به\nf دست آوردن این کلید، یکی از دو دستور زیر باید اجرا شوند. پس از اجرای یکی از دستورات زیر، مشکل رفع خواهدشد.
\begin{latin}
\setlength{\parindent}{0ex}
\renewcommand{\labelenumi}{\Roman{enumi}}
\begin{enumerate}
\item) gpg -\nf -keyserver hkp://keys.gnupg.net -\nf -recv-keys 409B6B1796C275462A1703

113804BB82D39DC0E37D2BAF1CF37B13E2069D6956105BD0E739499BDB 
\item) curl -sSL https://rvm.io/mpapis.asc | gpg -\nf -import -
\end{enumerate}
\end{latin}

\subsubsection{دستور شماره ۳}
\begin{latin}
\setlength{\parindent}{0ex}
3. source \char`\~/.rvm/scripts/rvm
\end{latin}
برخی از اسکریپت\nf های \lr{bash}، به\nf جای این\nf که توسّط یک دستور اجرایی\RTLfootnote{\lr{Executable command}}، اجرا شوند، باید توسّط \lr{syntax} زیر اجرا شوند. در دستور فوق، به جای عبارت \lr{<the script name>}، باید مکان قرارگیری اسکریپت و نام آن را قرار داد.
 \begin{latin}
\setlength{\parindent}{0ex}
source  <the script name>
\end{latin}
مشکلی که امکان دارد در اجرای دستور شماره \lr{3} به\nf وجود آید، تفاوت در پوشه\nf ی نصب \lr{rvm} است. برای رفع این مشکل، می\nf توانید آدرس پوشه\nf ی موردنظر را به\nf جای علامت  \char`\~ قرار دهید. راه دیگر این است که خط فرمان در هنگام اجرای دستورات \lr{1} و \lr{2}، در پوشه\nf ی  خانه قرار داشته باشد. آدرس این پوشه، به صورت \char`\~ و یا \lr{/home/ubuntu} است.  

%این است که مکان اسکریپت مورد نظر برای اجرای \lr{RVM}، با آنچه که مستندات \lr{clearwater} درنظر گرفته است، متفاوت می\nf باشد. بنابر مستندات \lr{clearwater}، این دستور در پوشه \lr{/.rvm/scripts} قرار دارد امّا با بررسی\nf های انجام شده، معلوم شد که این اسکریپت در پوشه \lr{/usr/local/rvm/scripts} و یا در پوشه \lr{/usr/local/rvm/src/rvm/scripts} قرار دارد. بنابراین، به\nf جای دستور شماره ۳، باید یک از دو دستور زیر اجرا شود. 
%\begin{latin}
%\setlength{\parindent}{0ex}
%\renewcommand{\labelenumi}{\Roman{enumi}}
%\begin{enumerate}
%\item) source /usr/local/rvm/scripts/rvm
%\item) source /usr/local/rvm/src/rvm/scripts/rvm
%\end{enumerate}
%\end{latin}

\subsubsection{دستور شماره 4}
\begin{latin}
\setlength{\parindent}{0ex}
4. rvm autolibs enable
\end{latin}
دستور فوق، \lr{autolibs} را فعّال می\nf کند. \lr{autolibs}، یک ویژگی پیاده\nf سازی\nf شده در \lr{RVM} است که امکان نصب خودکار یک سری وابستگی\nf ها را بر روی سیستم فراهم می\nf کند. این وابستگی\nf ها، معمولاً برنامه\nf هایی از قبیل \lr{OpenSSL} و \lr{YAML} می\nf باشند.

\subsubsection{دستور شماره ۵ و ۶}
\begin{latin}
\setlength{\parindent}{0ex}
5. rvm install 1.9.3

6. rvm use 1.9.3
\end{latin}
برای اجرای تست زنده، نیاز است که نسخه \lr{1.9.3} زبان \lr{Ruby} بر روی سیستم نصب شده باشد. دستور شماره \lr{5}، با استفاده از \lr{RVM} این کار را انجام می\nf دهد. دستور شماره \lr{6} نیز به خط فرمان دستور می\nf دهد که به\nf صورت پیش\nf فرض، از این نسخه\nf ی \lr{Ruby} استفاده شود.



\subsubsection{دستور شماره ۷}
دستور زیر،کدهای تست زنده\nf ی \lr{clearwater} را از \lr{git} دریافت می\nf کند.
\begin{latin}
\setlength{\parindent}{0ex}
7. git clone -b stable -\nf -recursive git@github.com:Metaswitch/clearwater-live-test.git
\end{latin}

\subsubsection{دستور شماره ۸ و ۹}
\begin{latin}
\setlength{\parindent}{0ex}
8. cd /clearwater-live-test

9. bundle install
\end{latin}
احتمالاً پس از اجرای دستور شماره 9، پیغام خطای \lr{Could not find quaff-0.7.4 in any of the sources} نمایش داده شود. برای رفع این مشکل باید دستور زیر را اجرا کرد.
\begin{latin}
\setlength{\parindent}{0ex}
git install quaff -v 0.7.4
\end{latin}